%% Template article for Elsevier's document class `elsarticle'
%% with numbered style bibliographic references
\documentclass[preprint,12pt]{elsarticle}

% remove preprint footnote
\makeatletter
\def\ps@pprintTitle{%
 \let\@oddhead\@empty
 \let\@evenhead\@empty
 \def\@oddfoot{\centerline{\thepage}}%
 \let\@evenfoot\@oddfoot}
\makeatother


\usepackage{setspace}
\doublespacing

\usepackage{hyperref} % auto detect ref type
\usepackage{natbib}
\setcitestyle{authoryear}

%% Use the option review to obtain double line spacing
%% \documentclass[preprint,review,12pt]{elsarticle}

%% Use the options 1p,twocolumn; 3p; 3p,twocolumn; 5p; or 5p,twocolumn
%% for a journal layout:
%% \documentclass[final,1p,times]{elsarticle}
%% \documentclass[final,1p,times,twocolumn]{elsarticle}
%% \documentclass[final,3p,times]{elsarticle}
%% \documentclass[final,3p,times,twocolumn]{elsarticle}
%% \documentclass[final,5p,times]{elsarticle}
%% \documentclass[final,5p,times,twocolumn]{elsarticle}

\usepackage{graphicx}
\usepackage{amssymb}
\usepackage{amsmath}
\usepackage{enumerate}
\usepackage{caption}
\usepackage{textcomp}    % for Hawaii characters
%\usepackage{listings}
%\lstset{language=R}
\usepackage[T1]{fontenc}  % tilde in middle in lstlisting
\usepackage[formats]{listings}
\lstset{language=R,
		basicstyle=\ttfamily,
		columns=fixed,
		basewidth=0.5em,
		literate={~}{{$\sim$}}1
		}

\biboptions{comma,round}

% shortcuts
\newcommand{\bbeta}{\boldsymbol{\beta}}
\newcommand{\blambda}{\boldsymbol{\lambda}}
\newcommand{\T}{\intercal}
\newcommand{\bS}{\mathbf{S}}
\newcommand{\bQ}{\mathbf{Q}}
\newcommand{\bSigma}{\boldsymbol{\Sigma}}
\newcommand{\bm}{\boldsymbol}  % bold maths symbols
\newcommand{\tl}{\tilde{\lambda}}   % thinned little lambda
\newcommand{\tL}{\tilde{\Lambda}}  % thinned big lambda

% RJC 09/08/2019 Added shortcuts for Hawaiian words
\newcommand{\akepa}{\textquotesingle\={a}kepa}  % adds Hawaiian diacritical marks
\newcommand{\Akepa}{\textquotesingle\={A}kepa}  % adds Hawaiian diacritical marks
\newcommand{\hawaii}{Hawai\textquotesingle i}   % adds Hawaiian diacritical marks
\DeclareMathOperator*{\argmax}{arg\,max}  % * means _ puts thing beneath operator

\begin{document}
\begin{frontmatter}
\title{One-stage point transect distance sampling using iterated integrated nested Laplace approximations}

%% use the tnoteref command within \title for footnotes;
%% use the tnotetext command for the associated footnote;
%% use the fnref command within \author or \address for footnotes;
%% use the fntext command for the associated footnote;
%% use the corref command within \author for corresponding author footnotes;
%% use the cortext command for the associated footnote;
%% use the ead command for the email address,
%% and the form \ead[url] for the home page:
%%
%% \title{Title\tnoteref{label1}}
%% \tnotetext[label1]{}
%% \author{Name\corref{cor1}\fnref{label2}}
%% \ead{email address}
%% \ead[url]{home page}
%% \fntext[label2]{}
%% \cortext[cor1]{}
%% \address{Address\fnref{label3}}
%% \fntext[label3]{}


%% use optional labels to link authors explicitly to addresses:
%% \author[label1,label2]{<author name>}
%% \address[label1]{<address>}
%% \address[label,2]{<address>}

% RJC 09/08/2019 Added co-authors and affiliations. Note that I used \affil[]{} instead of \address[]{}
\author[1,*]{Andrew E Seaton}
\author[1,2]{Richard J Camp}
\author[3]{Finn Lindgren}
\author[1]{Janine B Illian}
\author[1]{David L Borchers}
\author[1]{David L Miller}      % Ask about co-authoring the manuscript
\author[1]{Len Thomas}          % Ask about co-authoring the manuscript
\author[1]{Stephen T Buckland}  % Ask about co-authoring the manuscript
\author[4]{Steve J Kendall}     % It is Steve's data we are using and I have already mentioned this manuscript to him.

%\address{Centre for Research into Ecological \& Environmental Modelling and School of Mathematics \& Statistics, University of St Andrews, St Andrews, Fife, Scotland}
\address[1]{Centre for Research into Ecological \& Environmental Modelling and School of Mathematics \& Statistics, University of St Andrews, St Andrews, Fife, Scotland}
\address[2]{U. S. Geological Survey, Pacific Island Ecosystems Research Center, P.O. Box 44, \hawaii{} National Park, HI 96718, U.S.A.}
\address[3]{School of Mathematics, University of Edinburgh, Edinburgh, Scotland}
\address[4]{U. S. Fish and Wildlife, Big Island National Wildlife Refuge Complex, 60 Nowelo St., Suite 100, Hilo, HI  96720, U.S.A.}
\address[*]{Correspondence: Andrew E Seaton, Email: aes22@st-andrews.ac.uk}

\begin{abstract}
A New Abstract Goes Here   
\end{abstract}

%\begin{keyword}
%Distance sampling \sep Stochastic partial differential equations \sep Integrated nested Laplace approximation \sep Generalized additive model
%\end{keyword}
\end{frontmatter}

\section*{Introduction}

\begin{itemize}
	\item Estimating population size and spatial distribution is an important question in ecology and conservation
	\item Here we present the Akepa data:
		\begin{itemize}
			\item endangered
			\item subject of conservation efforts
			\item require monitoring to understand changes in overall abundance and the spatial distribution to target conservation efforts
		\end{itemize}
	\item Estimating the population size and spatial distribution of Akepa is challenging because
		\begin{enumerate}[(i)]
			\item cannot do a census (large numbers, dense forest, logistical constraints) and so must subsample in space and time (describe the HFBS design)
			\item imperfect detection when we do sample - a complex observation process (describe point transect distance sampling approach)
			\item interest in generating spatial predictions but we usually have missing covariates and use spatial random effects (No covariate for north-south in Akepa data, other studies ambiguous)
			\item complex models that are difficult to communicate with stakeholders such as conservation managers and policy officers (communicating uncertainty in abundance estimation and prediction maps)
		\end{enumerate}
\end{itemize}

In this paper we present an analysis of the Akepa data that seeks to address these issues

\begin{enumerate}[(i)]
	\item a spatial point process formulation of distance sampling
	\item a one-stage estimation process to simultaneously estimate the spatial model and the observation process
	\item a computationally efficient Gaussian Markov random field effect based on a stochastic partial differential equation approach
	\item model evaluation and communication based on join posterior sampling and excursions package
\end{enumerate}

We work in a Bayesian context here but note that maximum-likelihood frameworks could also be used.

\section{Study design}

\begin{itemize}
	\item describe survey protocol in detail
	\item mini lit review - DS, 2-stage, Joyce
\end{itemize}

\section{Distance Sampling as a Thinned Point Process}

\section{Iterated INLA}

\begin{itemize}
	\item mini lit review - communicating uncertainty in spatial models w random effects
	(should this really be here??  Think about moving)
\end{itemize}

\section{GMRF + SPDE}

\section{Results}

\section{Model Evaluation}

\begin{itemize}
	\item Posterior sampling
	\item Barplot for random effect
	\item Excursions
\end{itemize}

\section{Discussion}

Discuss wisely

\end{document}